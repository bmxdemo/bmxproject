\documentclass{article}

\begin{document}
\title{Understanding Stoke's parameters}
\author{A Slosar}
\maketitle
\section{Four numbers}

The right way to think about polarization states is to think of the
system as being fully described by two complex numbers at any one
point in time. For incoherent light these complex numbers ``flicker'',
but they have well defined averages and cross averages that give the
Stokes parameters.

Let's take the two numbers to be the electric field in the $\hat{x}$
and $\hat{y}$ directions:
\begin{eqnarray}
  E_x &=& X = a + ib \\
  E_x &=& Y = c + id 
\end{eqnarray}
The 45 degree rotated states are simply given by 
\begin{eqnarray}
  A &=& \frac{1}{\sqrt{2}} \left(X+Y\right)\\
  B &=& \frac{1}{\sqrt{2}} \left(X-Y\right)
\end{eqnarray}
and the left and right cirular polarisations by 
\begin{eqnarray}
  L &=& \frac{1}{\sqrt{2}} \left(X+iY\right)\\
  R &=& \frac{1}{\sqrt{2}} \left(X-iY\right)
\end{eqnarray}

The Stokes parameters are defined to be
\begin{eqnarray}
  U&=&\left<|X^2| + |Y^2| \right> \\
  Q&=&\left<|X^2| - |Y^2| \right> \\
  U&=&\left<|A^2| - |B^2| \right>\\
  V&=&\left<|L^2| - |R^2|\right> 
\end{eqnarray}
In any one description, we have four numbers, although the correct way
to think about this is that at the level $X$ and $Y$ (or equivalently
$A$,$B$ or $L$,$R$) these are really electric fields with some phases,
while Stokes parameters are time averages.

\section{Given $X$ and $Y$}
A simple complex algebra tells us

\begin{eqnarray}
  \left<X X^*\right> &=& U+Q\\
  \left<Y Y^*\right> &=& U-Q\\
  \left<X Y^*\right> &=& U-iV\\
\end{eqnarray}

\section{Given $L$ and $R$}
A simple complex algebra tells us

\begin{eqnarray}
  \left<L L^*\right> &=& I+V\\
  \left<R R^*\right> &=& I-V\\
  \left<L R^*\right> &=& Q+iU\\
\end{eqnarray}


\end{document}
