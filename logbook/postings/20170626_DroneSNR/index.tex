\documentclass{article}
\usepackage{gensymb}

\begin{document}

\title{Drone SNR calculation}
\author{A Slosar}
\date{June 26, 2017}
\maketitle

\newcommand{\Dnu}{\Delta \nu}
Assume drone flies at heigh $h\approx150$m, emits power $P_e$ into
bandwidth $\Dnu$ over solid angle
$\Omega_e\approx 2\pi$, which is received by a dish of radius $r_d=2$
and is beam suppressed by $A=10^{-3}-1$ (assuming we want to measure
beam to -30dB).  Received signal power is given by

\begin{equation}
  P_S = P_e A \frac{\pi r_d^2}{h^2\Omega_e} \approx 10^{-4} P_e A
\end{equation}

The noise power is given by sum over system $T_{\rm sys}\approx 50$K
and sky $T_{\rm sky} \approx 10$K temperatures, giving total
temperature $T_T\approx 60$K

\begin{equation}
  P_N = 2 k_B T_T \Dnu
\end{equation}
We want signal to be always dominated by drone, rather than the
changing sky background, etc, which implies $P_S \gg P_N$, or 

\begin{equation}
  \Dnu \ll A \left(\frac{P_e}{\rm 1  mW}\right) \, 6\times 10^{13} {\rm Hz} 
\end{equation}

So, we are always very safely into this regime up to $A=10^{-4}$

Now say we want to map a square of $30\degree\times30\degree$
with resolution $0.1\degree$. This gives $N_{\rm pix} = 10^5$
pixels. Flying for $T=10$ minutes, this gives integration time
\begin{equation}
  t_i = \frac{T}{N_{\rm pix}}\approx 0.4{\rm s}
\end{equation}
per resolution element.

Given that we have established that noise never matters, we have SNR
that is purely ``mode counting''

\begin{equation}
  {\rm SNR} = \frac{P_S}{(P_S + P_N) / \sqrt{t_i \Dnu }} \approx \sqrt{t_i \Dnu}
\end{equation}

This is somewhat counter-intuitive, because I would have thought that
for a delta function $\Dnu$ I should have meausured the amplitude
perfectly. So I think this indicates breakdown of the ``Gaussian
field'' approximation. So, let's do two calculations:

\subsection*{Broadband signal}

With $\Dnu=500$MHz, $P_N\approx 10^{-12}$W and $P_S=A\times10^{-7}$W for mW
source, giving

\begin{equation}
  {\rm SNR} = 10^4 \frac{A}{A+10^{-5}}
\end{equation}
So essentially a very good SNR down to -50dB. Emitting 10$\mu$W would
give us signal to -30dB.


\subsection*{Narrowband signal}

Say you have a narrowband signal, a pure tone with infinite
precision that goes into a single FFT bin. The signal value in volts
in that FFT bin after integration $t_i$ is going to be $V=\sqrt{2 P_s
A  R}$, where $R=50\Omega$ is the line impendance.

At the same time the thermal noise injects $2k_B T_T / t_i$  power
into it (since the width of the FFT bin is $\sim 1/t_i$), giving rms
noise contribution of $\sigma(V_n) \sim  \sqrt{2 R k_B T_t /t_i}$

The ratio gives SNR
\begin{equation}
  {\rm SNR} = \sqrt{\frac{P_s A}{k_BT_T} t_i} = 4\times 10^8 \sqrt{A}
  \sqrt{\frac{P_s}{1{\rm mW}}}
\end{equation}

If this calculation is right, which I'm not sure it is, then the
scaling is very favorable and and we emit very very small powers and
still get away with it, even for much decreased $t_i$ and deep into
small values of $A$.

Note that scaling with $A$ is more favorable. This squares with my
intuition that if measuring amplitude of a known signal is much easier
than measuring ``power''.

\end{document}
